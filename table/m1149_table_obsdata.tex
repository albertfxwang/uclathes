% = = = = = = = = = = = = = = = = = = = = = = = = = = = = = = = = = = = = = = = = = =
% Include this table with \input{filename.tex}
% To rotate in emulateapj do: \begin{turnpage}\input{filename.tex}\end{turnpage}
% To display it on multiple pages do: \LongTables\input{filename.tex}
% - - - - - - - - - - - - - - - - - - - - - - - - - - - - - - - - - - - - - - - - - -
{\setlength\tabcolsep{2pt}
\begin{deluxetable}{cccp{2.5cm}ccccccccc} \tablecolumns{13}
\tablewidth{0pt}
\tablecaption{Properties of the grism spectroscopic data used in this work}
% - - - - - - - - - - - - - - - - - - - - - - - - - - - - - - - - - - - - - - - - - -
\tablehead{
    \colhead{PA\tablenotemark{a}} &
    \colhead{Grism} &
    \colhead{Exposure Time} &
    \colhead{Program}  &
    \colhead{Time of completion}\\
    \colhead{(deg.)} & & 
    \colhead{(s)} & & &
}
%---------------------------------------------------------------
\startdata
\multirow{2}{*}{032}    & G102 & 8723 & \glass & \multirow{2}{*}{Feburary 2014} \\
                        & G141 & 4412 & \glass  \\
111                     & G141 & 36088 & SN Refsdal follow-up & December 2014 \\
119                     & G141 & 36088 & SN Refsdal follow-up & January 2015 \\
\multirow{2}{*}{125}    & G102 & 8623 & \glass & \multirow{2}{*}{November 2014} \\
                        & G141 & 4412 & \glass
\enddata
% - - - - - - - - - - - - - - - - - - - - - - - - - - - - - - - - - - - - - - - - - -
\tablecomments{Here we only include the grism observations targetted on the prime field of
\clsan.}
\tablenotetext{a}{The position angle shown here corresponds to the ``PA\_V3'' value
    reported in the WFC3/IR image headers. The position angle of the dispersion axis of the
grism spectra is given by $\mathrm{PA_{disp}} \approx \mathrm{PA\_V3} - 45$.}
\label{tab:obsdata}
\end{deluxetable}
}
