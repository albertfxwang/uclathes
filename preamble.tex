%%<<140818>> update this preamble file in terms of the beamer version
%%<<141022>> merged the beamer version and the manu version together and re-locate it at latex_refs/
%%<<190330>> adapted for my thesis. Need to add the following line to bibtex. see: https://tex.stackexchange.com/questions/365215/natbib-newblock-undefined-error-with-informs3-document-class
\newcommand{\newblock}{}

%-------------- declare old font command, see
%               https://tex.stackexchange.com/questions/304311/is-there-any-reason-not-to-use-let-to-redefine-a-deprecated-control-sequence-to
\DeclareOldFontCommand{\rm}{\normalfont\rmfamily}{\mathrm}
\DeclareOldFontCommand{\sf}{\normalfont\sffamily}{\mathsf}
\DeclareOldFontCommand{\tt}{\normalfont\ttfamily}{\mathtt}
\DeclareOldFontCommand{\bf}{\normalfont\bfseries}{\mathbf}
\DeclareOldFontCommand{\it}{\normalfont\itshape}{\mathit}
\DeclareOldFontCommand{\sl}{\normalfont\slshape}{\@nomath\sl}
\DeclareOldFontCommand{\sc}{\normalfont\scshape}{\@nomath\sc}


\newcommand{\NimgTOT}{179}   %number of TOTal identified arc images
\newcommand{\NsysTOT}{57}   %number of TOTal identified arc systems
\newcommand{\NimgUSE}{72}   %number of USEd in model arc images
\newcommand{\NsysUSE}{25}   %number of USEd in model arc systems
\newcommand{\NimgELtot}{7}  %number of arc images showing emission lines with quality > 1
\newcommand{\NsysELtot}{5}   %number of arc systems showing emission lines with quality > 1
\newcommand{\NimgELhiQ}{5}   %number of arc images showing high-quality (3 or 4) emission lines  <=>  z_spec confirmed
\newcommand{\NsysELhiQ}{3}   %number of arc systems showing high-quality (3 or 4) emission lines  <=>  z_spec confirmed
\newcommand{\NELobjsQhi}{55}    %number of all the ELobjs with quality 3 or 4, including the \NimgELhiQ{} arcs
\newcommand{\NELQone}{2}        %number of singly lensed ELobjs with quality 1
\newcommand{\NELQtwo}{18}       %number of singly lensed ELobjs with quality 2
\newcommand{\NELQthree}{16}     %number of singly lensed ELobjs with quality 3
\newcommand{\NELQfour}{34}      %number of singly lensed ELobjs with quality 4


%%%%%%%%%%%%%%%%%%%%%%%%%%%%%%%%%%%%%%%%%%%%%%
%%%% packages and settings useful globally
%%%%%%%%%%%%%%%%%%%%%%%%%%%%%%%%%%%%%%%%%%%%%%
\usepackage[english]{babel} % default (American) English hyphenation
\usepackage[utf8]{inputenc} % useful to type directly diacritic characters
\usepackage[T1]{fontenc}    % Use vector fonts, so it zooms properly in on-screen viewing software
\usepackage{ae,aecompl}
\usepackage{graphicx}       % allows the usage of ``includegraphics''
\usepackage{natbib}         % enables the use of \citep and others; see http://merkel.zoneo.net/Latex/natbib.php
\usepackage{url}            % enables the use of url web links
\urlstyle{rm}
\usepackage{grffile}        % enables dots and underscores in .pdf filenames
\usepackage{mathtools}      % enables the use of various environments work, e.g., |align|, |dcases|, define symbols (see below), etc.
\usepackage{bbold}          % enables the use of \mathbb{1} as the identity matrix
\newcommand{\bb}{\mathbb}
\usepackage{bbm}            % enables the use of bold Greek symbols
\newcommand{\bs}{\boldsymbol}
%\DeclareGraphicsRule{.tif}{png}{.jpg}{.bmp}
\usepackage{multirow}       % enables the use of \multirow{num}{*}{text} in deluxetable
\usepackage{xspace}         % enables the use of \xspace in defining macros
%\usepackage{epstopdf}
\usepackage{ccaption}       % continuation of caption for figures containing multiple floats
\usepackage{amsmath,amssymb,amsxtra,amsfonts}   % to use pmatrix, etc.
\usepackage{txfonts}
\usepackage{deluxetable}
\usepackage{lscape}

%%%%%%%%%%%%%%%%%%%%%%%%%%%%%%%%%%%%%%%%%%%%%%
%%%% special packages and corresponding usages
%%%%%%%%%%%%%%%%%%%%%%%%%%%%%%%%%%%%%%%%%%%%%%
%-------------- provide circled numbers
%\usepackage{tikz}
%\newcommand*\circled[1]{\tikz[baseline=(char.base)]{\node[shape=circle,draw,inner sep=2pt] (char) {#1};}}
%-------------- use \mathsmaller and \mathlarger to resize math symbols
%\usepackage{relsize}
%-------------- for dynamic display of tables, see
%               http://www.math-linux.com/latex-26/article/how-to-make-a-presentation-with-latex-introduction-to-beamer
%\usepackage{colortbl}
%-------------- correct total number of pages for appendix, have to be commented off when using \appendix in ms.tex
%               http://texblog.org/2012/05/24/correct-total-number-of-pages-with-backup-slides-in-beamer-presentations/
%\usepackage{appendixnumberbeamer}
%-------------- strikethrough in beamer itemize, see
%               http://tex.stackexchange.com/questions/35767/in-beamer-how-to-strike-through-an-item-after-displaying
%\usepackage{ulem}       % turn \emph to \underline
%\renewcommand<>{\sout}[1]{\alt#2{\beameroriginal{\sout}{#1}}{#1}}
%-------------- enable the use of enumitem (e.g. for creating checkbox todo list, see below), see
%				http://tex.stackexchange.com/questions/24371/does-enumitem-conflict-with-beamer-for-lists
%\usepackage{enumitem}      %  NOTE: once uncommented, cannot use env {itemize, enumerate} any more
%\setitemize{label=\usebeamerfont*{itemize item}%
%  \usebeamercolor[fg]{itemize item}
%  \usebeamertemplate{itemize item}}
%%-------------- create checkbox and todo list in itemize (need to enable enumitem), see
%%               http://tex.stackexchange.com/questions/247681/how-to-create-checkbox-todo-list
%\usepackage{enumitem}      %  NOTE: once uncommented, cannot use env {itemize, enumerate} any more
%\newlist{todolist}{itemize}{2}
%\setlist[todolist]{label=$\square$}
%\usepackage{pifont}
%\usepackage{amsmath}
%\usepackage{amssymb}
%\newcommand{\cmark}{\ding{51}}%
%\newcommand{\xmark}{\ding{55}}%
%\newcommand{\done}{\rlap{$\square$}{\raisebox{2pt}{\large\hspace{1pt}\cmark}}%
%\hspace{-2.5pt}}
%\newcommand{\wontfix}{\rlap{$\square$}{\large\hspace{1pt}\xmark}}
%-------------- include videos, works for .flv, .mov format
%\usepackage{media9}
%-------------- encourage a break at this point
%\newcommand{\smallspace}{\vspace{3cm}\goodbreak}
%\newcommand{\bigspace}{\vspace{4.5cm}\goodbreak}
%\newcommand{\biggerspace}{\vspace{5.5cm}\goodbreak}
%-------------- put the points in the left margin, see latex_manuscripts/reference/hw3_sol_fin.tex
%\reversemarginpar
%\newcommand{\worth}[1]
%        {\mbox{}\marginpar{\em #1}\nolinebreak}%puts value in margin
%-------------- Flexible typesetting of table and figure floats, see: https://ctan.org/pkg/ctable
%\usepackage{ctable}
%-------------- add texts in box, see: https://tex.stackexchange.com/questions/36524/how-to-put-a-framed-box-around-text-math-environment/36528
%\usepackage{tcolorbox}
%-------------- define new column type, see: https://tex.stackexchange.com/questions/257128/how-does-the-newcolumntype-command-work
%\usepackage{array}
%\newcolumntype{L}[1]{>{\raggedright\let\newline\\\arraybackslash\hspace{5pt}}m{#1}}

% -----------------------------------------------------------------------------------------------
% Turning references and citations into hyperlinks in color
\usepackage{color}         % using color package; put color boxes around text: \fcolorbox{frame colour}{background colour}{text}
\definecolor{gold}{rgb}{1,0.80,0}
\definecolor{orange}{rgb}{1,0.5,0}
\definecolor{midgray}{gray}{0.3}
\definecolor{lblue}{rgb}{0,0.2,0.6}
\definecolor{dgreen}{rgb}{0.1,0.6,0.3}
\definecolor{purple}{rgb}{0.5019607843137255,0.0,0.5019607843137255}
\usepackage[colorlinks=true,citecolor=lblue,linkcolor=black]{hyperref}    % does not work when enabling draft, nor with beamer

% -----------------------------------------------------------------------------------------------
% Reference a section by ``section XX'', without generating text like ``subsection XX'' as is done by \autoref.
\newcommand{\secref}[1]{Section~\ref{#1}}
\newcommand{\figref}[1]{Figure~\ref{#1}}
\newcommand{\Eqref}[1]{Equation~(\ref{#1})}


%%%%%%%%%%%%%%%%%%%%%%%%%%%%%%%%%%%%%%%%%%%%%%
%%%% only necessary for poster
%%%%%%%%%%%%%%%%%%%%%%%%%%%%%%%%%%%%%%%%%%%%%%
%\usepackage{wrapfig}
%\usepackage{caption}

%%%%%%%%%%%%%%%%%%%%%%%%%%%%%%%%%%%%%%%%%%%%%%
%%%% only necessary for emulateapj
%%%%%%%%%%%%%%%%%%%%%%%%%%%%%%%%%%%%%%%%%%%%%%
% - - - - - - - - already defined
\newcommand\farcs{\mbox{$.\!^{\prime\prime}$}}    % arcsec with a decimal point in front
\newcommand\farcm{\mbox{$.\mkern-4mu^\prime$}}      % arcmin with a decimal point in front
\newcommand{\arcsec}{\mbox{$\!^{\prime\prime}$}\xspace}  % the unit of arcsec
\newcommand{\arcmin}{\mbox{$\!^{\prime}$}\xspace}          % the unit of arcmin
\newcommand{\arcdeg}{\mbox{$^{\circ}$}\xspace}             % the unit of arcdeg
% - - - - - - - - have to comment off when using emulateapj
\usepackage{marvosym}


%%%%%%%%%%%%%%%%%%%%%%%%%%%%%%%%%%%%%%%%%%%%%%
%%%% mathematical symbols
%%%%%%%%%%%%%%%%%%%%%%%%%%%%%%%%%%%%%%%%%%%%%%
%\font\fb=''[cmr10]'' %for use with \LaTeX command
% pre-existing symbols: \exp, \ln, \log
%\newcommand{\rmd}{\textrm{d}}
%\def\d{\textrm{d}}   % NOTE: \newcommad doesn't work but this works!
\def\prt{\partial}
\def\cP{{\cal P}}
\newcommand{\be}{\begin{equation}}
\newcommand{\ee}{\end{equation}}
\newcommand{\non}{\nonumber}
\newcommand{\ba}{\begin{align}}
\newcommand{\ea}{\end{align}}
\newcommand{\nt}{\notag}
\def\abs#1{\left|{#1}\right|}
\newcommand{\avg}[1]{\left<#1\right>}
%-------- provide define symbols
\newcommand{\defeq}{\vcentcolon=}
\newcommand{\eqdef}{=\vcentcolon}
%-------- formatted arrows
\newcommand{\Ra}{\ensuremath{\Rightarrow}\xspace}
\newcommand{\ra}{\ensuremath{\rightarrow}\xpace}
\newcommand{\lra}{\ensuremath{\Leftrightarrow}\xspace}
% Statistical mean (angle brackets)
\newcommand{\stmean}[1]{\langle{#1}\rangle}
\newcommand{\logg}{\log_{10}}	% base-10 logarithm
% Spatial-curvature function in cosmological distances.
\DeclareMathOperator{\sk}{S}
% Trace of a linear operator or matrix
\DeclareMathOperator{\tr}{tr}
% Differential (must be upright Roman letter, per MNRAS)
\DeclareMathOperator{\ud}{d}

%%%%%%%%%%%%%%%%%%%%%%%%%%%%%%%%%%%%%%%%%%%%%%
%%%% (astro)physical units and quantities
%%%%%%%%%%%%%%%%%%%%%%%%%%%%%%%%%%%%%%%%%%%%%%
% - - - - - - - - quantities
\newcommand{\Msun}{\ensuremath{M_\odot}\xspace}
\newcommand{\Rsun}{\ensuremath{R_\odot}\xspace}
\newcommand{\Lsun}{\ensuremath{L_\odot}\xspace}
\newcommand{\thE}{\ensuremath{\theta_{\rm E}}\xspace}
\newcommand{\chisq}{\ensuremath{\chi^2}\xspace}
\newcommand{\zspec}{\ensuremath{z_{\rm spec}}\xspace}
\newcommand{\zphot}{\ensuremath{z_{\rm phot}}\xspace}
\newcommand{\Mstar}{\ensuremath{M_\ast}\xspace}
\newcommand{\Lstar}{\ensuremath{L_\ast}\xspace}
\newcommand{\Sstar}{\ensuremath{\Sigma_\ast}\xspace}
\newcommand{\oh}{\ensuremath{12+\log({\rm O/H})}\xspace}
\newcommand{\Av}{\ensuremath{A_{\rm V}}\xspace}
\newcommand{\Rv}{\ensuremath{R_{\rm V}}\xspace}
\newcommand{\Te}{\ensuremath{T_{\rm e}}\xspace}
\def\ne{\ensuremath{n_{\rm e}}\xspace}
\newcommand{\SFR}{\ensuremath{{\rm SFR}}\xspace}
\newcommand{\Mgas}{\ensuremath{M_{\rm gas}}\xspace}
\newcommand{\Sgas}{\ensuremath{\Sigma_{\rm gas}}\xspace}
\newcommand{\fgas}{\ensuremath{f_{\rm gas}}\xspace}
\newcommand{\Zgas}{\ensuremath{Z_{\rm gas}}\xspace}
\newcommand{\tage}{\ensuremath{t_{\rm age}}\xspace}
\newcommand{\Vrot}{\ensuremath{V_{\rm rot}}\xspace}
\newcommand{\reff}{\ensuremath{r_{\rm eff}}\xspace}
\newcommand{\Dn}{\ensuremath{{\rm D}_n(4000)}\xspace}
\newcommand{\HdA}{\ensuremath{{\rm H}\delta_A}\xspace}
\newcommand{\scrit}{\ensuremath{\sigma_{\rm crit}}\xspace}

% - - - - - - - - units
\newcommand{\eV}{\ensuremath{\rm eV}\xspace}
\newcommand{\pc}{\ensuremath{\rm pc}\xspace}
\newcommand{\kpc}{\ensuremath{\rm kpc}\xspace}
\newcommand{\Mpc}{\ensuremath{\rm Mpc}\xspace}
\newcommand{\K}{\ensuremath{\rm K}\xspace}
\newcommand{\mK}{\ensuremath{\rm mK}\xspace}
\newcommand{\Hunit}{\ensuremath{\rm km~s^{-1}~Mpc^{-1}}\xspace}
\newcommand{\Funit}{\ensuremath{\rm erg~s^{-1}~cm^{-2}}\xspace}
\newcommand{\Flam}{\ensuremath{\rm erg~s^{-1}~cm^{-2}~\AA^{-1}}\xspace}
\newcommand{\Fnu}{\ensuremath{\rm erg~s^{-1}~cm^{-2}~Hz^{-1}}\xspace}
\newcommand{\muJy}{\ensuremath{\mu\rm Jy}\xspace}
\newcommand{\SBunit}{\ensuremath{\rm erg~s^{-1}~cm^{-2}~arcsec^{-2}}\xspace}
\newcommand{\magarcs}{\ensuremath{\rm mag~arcsec^{-2}}\xspace}
\newcommand{\Msunyr}{\ensuremath{\Msun~\mathrm{yr}^{-1}}\xspace}
\newcommand{\yr}{\ensuremath{\rm yr}\xspace}
\newcommand{\Myr}{\ensuremath{\rm Myr}\xspace}
\newcommand{\Gyr}{\ensuremath{\rm Gyr}\xspace}
\def\micron{\ensuremath{\mu\textrm{m}}\xspace}  % better than the default \micron, which does not use \xspace
\newcommand{\kms}{\ensuremath{\rm km~s^{-1}}\xspace}

%%%%%%%%%%%%%%%%%%%%%%%%%%%%%%%%%%%%%%%%%%%%%%
%%%% astrophysical aliases and jargons
%%%%%%%%%%%%%%%%%%%%%%%%%%%%%%%%%%%%%%%%%%%%%%
\newcommand\ionp[2]{#1$\;${\scshape{#2}}}      % ion permitted transitions, i.e., C IV = \ionp{C}{iv}
\newcommand\ionf[2]{[#1$\;${\scshape{#2}}]}    % ion forbidden transitions, i.e., [O III] = \ionf{O}{iii}
\newcommand\ions[2]{#1$\;${\scshape{#2}}]}     % ion semi-forbidden transitions, i.e., C III] = \ions{C}{iii}
\newcommand{\Ha}{\textrm{H}\ensuremath{\alpha}\xspace}
\newcommand{\Hb}{\textrm{H}\ensuremath{\beta}\xspace}
\newcommand{\Hg}{\textrm{H}\ensuremath{\gamma}\xspace}
\newcommand{\HII}{\textrm{H}\textsc{ii}\xspace}
\newcommand{\HI}{\textrm{H}\textsc{i}\xspace}
\newcommand{\Htwo}{\textrm{H}\ensuremath{_2}\xspace}
\newcommand{\He}{\textrm{He}\xspace}
\newcommand{\OI}{[\textrm{O}~\textsc{i}]\xspace}
\newcommand{\OII}{[\textrm{O}~\textsc{ii}]\xspace}
\newcommand{\OIII}{[\textrm{O}~\textsc{iii}]\xspace}
\newcommand{\CIII}{\textrm{C}~\textsc{iii}]\xspace}
\newcommand{\NII}{[\textrm{N}~\textsc{ii}]\xspace}
\newcommand{\SII}{[\textrm{S}~\textsc{ii}]\xspace}
\newcommand{\NeIII}{[\textrm{Ne}~\textsc{iii}]\xspace}
\newcommand{\sersic}{S\'{e}rsic\xspace}
\newcommand{\lya}{\textrm{Ly}\ensuremath{\alpha}\xspace}
% - - - - - - - - astrometric filters
\def\B{\ensuremath{B_{435}}\xspace}
\def\V{\ensuremath{V_{606}}\xspace}
\def\I{\ensuremath{I_{814}}\xspace}
\def\Y{\ensuremath{Y_{105}}\xspace}
\def\J{\ensuremath{J_{125}}\xspace}
\def\JH{\ensuremath{JH_{140}}\xspace}
\def\H{\ensuremath{H_{160}}\xspace}


%%%%%%%%%%%%%%%%%%%%%%%%%%%%%%%%%%%%%%%%%%%%%%
%%%% my specific macros for objects, software, instruments, telescopes, projects
%%%%%%%%%%%%%%%%%%%%%%%%%%%%%%%%%%%%%%%%%%%%%%
% - - - - - - - - celestial objects
\newcommand{\clyi}{MACS1149.6+2223\xspace}
\newcommand{\cler}{Abell 2744\xspace}
\newcommand{\clsan}{Abell 370\xspace}
\newcommand{\clsi}{MACS0416.1-2403\xspace}
\newcommand{\clwu}{MACS0717.5+3745\xspace}
\newcommand{\clliu}{RXJ2248.7-4431\xspace}
\newcommand{\clqi}{RXJ1347.5-1145\xspace}
\newcommand{\clba}{MACS0744.9+3927\xspace}
\newcommand{\cljiu}{MACS2129.4-0741\xspace}
\newcommand{\clshi}{MACS1423.8+2404\xspace}

% - - - - - - - - software
\newcommand{\pylf}{\textsc{pyLensFix}\xspace}
\newcommand{\lf}{\textsc{LensFix}\xspace}
\newcommand{\sw}{\textsc{SWunited}\xspace}
\newcommand{\sex}{\textsc{SExtractor}\xspace}
\newcommand{\emc}{\textsc{Emcee}\xspace}
\newcommand{\linmix}{\textsc{linmix}\xspace}
\newcommand{\adriz}{\textsc{AstroDrizzle}\xspace}
\newcommand{\dpac}{\textsc{DrizzlePac}\xspace}
\newcommand{\fast}{\textsc{FAST}\xspace}
\newcommand{\galfit}{\textsc{Galfit}\xspace}
\newcommand{\axe}{\textsc{aXe}\xspace}
\def\lt{\textsc{Lenstool}\xspace}
\newcommand{\glafic}{\textsc{Glafic}\xspace}
\newcommand{\gasoline}{\textsc{Gasoline}\xspace}
\newcommand{\ramses}{\textsc{Ramses}\xspace}
\newcommand{\SJ}{\textsc{Sharon \& Johnson}\xspace}
\newcommand{\grzl}{\textsc{Grzili}\xspace}
\newcommand{\burst}{\textsc{Starburst99}\xspace}

% - - - - - - - - projects, telescopes, instruments
\newcommand{\planck}{\textit{Planck}\xspace}
\newcommand{\hst}{\textit{HST}\xspace}
\newcommand{\jwst}{\textit{JWST}\xspace}
\newcommand{\spitzer}{\textit{Spitzer}\xspace}
\newcommand{\herschel}{\textit{Herschel}\xspace}
\newcommand{\chandra}{\textit{Chandra}\xspace}
\newcommand{\glass}{\textit{GLASS}\xspace}
\newcommand{\wisp}{\textit{WISP}\xspace}
\newcommand{\clash}{\textit{CLASH}\xspace}
\newcommand{\candels}{\textit{CANDELS}\xspace}
\newcommand{\hff}{\textit{HFF}\xspace}
\newcommand{\muse}{\textit{MUSE}\xspace}
\newcommand{\kmos}{\textit{KMOS}\xspace}
\newcommand{\keck}{\textit{Keck}\xspace}
\newcommand{\deimos}{\textit{DEIMOS}\xspace}
\newcommand{\mosfire}{\textit{MOSFIRE}\xspace}
\newcommand{\surfsup}{\textit{SURFSUP}\xspace}
\newcommand{\kd}{\textit{KMOS}$^{3\rm D}$\xspace}
\newcommand{\sdss}{\textit{SDSS}\xspace}
\def\clash{\textit{CLASH}\xspace}
\def\mosdef{\textit{MOSDEF}\xspace}
\newcommand{\vlt}{\textit{VLT}\xspace}
\newcommand{\osiris}{\textit{OSIRIS}\xspace}
\newcommand{\sinf}{\textit{SINFONI}\xspace}
\newcommand{\wfst}{\textit{WFIRST}\xspace}
\newcommand{\niriss}{\textit{NIRISS}\xspace}


%%%%%%%%%%%%%%%%%%%%%%%%%%%%%%%%%%%%%%%%%%%%%%
%%%% format, wording and abbreviations
%%%%%%%%%%%%%%%%%%%%%%%%%%%%%%%%%%%%%%%%%%%%%%
\def\etal{et al.\xspace}
\def\ie{i.e.\xspace}
\def\eg{e.g.\xspace}
\def\etc{etc.\xspace}
\def\aka{a.k.a.\xspace}
\def\vsv{vis-\'a-vis\xspace}
\renewcommand\({\left(}
\renewcommand\){\right)}
%\renewcommand\[{\left[}
%\renewcommand\]{\right]}

% - - - - - - - - word combo
\newcommand\mm{metallicity map\xspace}
\newcommand\mms{metallicity maps\xspace}
\newcommand\mg{metallicity gradient\xspace}
\newcommand\mgs{metallicity gradients\xspace}
\newcommand\Mgs{Metallicity gradients\xspace}
\newcommand\mgm{metallicity gradient measurement\xspace}
\newcommand\mgms{metallicity gradient measurements\xspace}
\newcommand\sr{spatially resolved\xspace}
\newcommand\srs{spatially resolved spectroscopy\xspace}
\newcommand\sra{spatially resolved analysis\xspace}
\newcommand\gp {gas-phase\xspace}
\newcommand\gpm{gas-phase metallicity\xspace}
\newcommand\subr{surface brightness\xspace}        % <<160715>> NOTE: cannot re-DEF \sb
\def\sf{star-forming\xspace}
\newcommand\sfr{star-formation rate\xspace}
\newcommand\sfh{star-formation history\xspace}
\newcommand\sfms{star-formation main sequence\xspace}

% - - - - - - - - specially formated words
\newcommand{\el}[1]{\ensuremath{\textrm{EL}_{#1}}}
\newcommand{\obs}{\textrm{o}}
\newcommand{\theo}{\textrm{t}}
\newcommand{\ext}{\textrm{ext}}
\def\det{\textrm{det}}
\newcommand\refe{\textrm{ref}}
\newcommand\pa{\textrm{PA}}

\def\p{{\rm prior}}
\def\fid{{\rm fid}}
\def\lnk{\kappa}
\def\lnkp{\kappa'}
%\newcommand{\n}{\noindent}


%%%%%%%%%%%%%%%%%%%%%%%%%%%%%%%%%%%%%%%%%%%%%%
%%%% lensing quantities and parameters
%%%%%%%%%%%%%%%%%%%%%%%%%%%%%%%%%%%%%%%%%%%%%%
\newcommand{\xa}{\alpha}
\newcommand{\xb}{\beta}
\newcommand{\xk}{\kappa}
\newcommand{\xg}{\gamma}
%\newcommand{\xg}[1]{|\gamma #1|}


%%%%%%%%%%%%%%%%%%%%%%%%%%%%%%%%%%%%%%%%%%%%%%
%%%% cosmological parameters
%%%%%%%%%%%%%%%%%%%%%%%%%%%%%%%%%%%%%%%%%%%%%%
\newcommand{\Or} {\ensuremath{\Omega_{\rm{r}}}\xspace}
\newcommand{\Om} {\ensuremath{\Omega_{\rm{m}}}\xspace}
\newcommand{\Ok} {\ensuremath{\Omega_{\rm{k}}}\xspace}
\newcommand{\Ol} {\ensuremath{\Omega_{\Lambda}}\xspace}
\newcommand{\Obh}{\ensuremath{\Omega_{\rm{b}}h^2}\xspace}
\newcommand{\Ob} {\ensuremath{\Omega_{\rm{b}}}\xspace}
\newcommand{\Onu}{\ensuremath{\Omega_\nu}\xspace}
\newcommand{\fnu}{\ensuremath{f_{\nu}}\xspace}
\newcommand{\Och}{\ensuremath{\Omega_{\rm{DM}}h^2}\xspace}
\newcommand{\Oc} {\ensuremath{\Omega_{\rm{DM}}}\xspace}
\newcommand{\ns} {\ensuremath{n_{\rm s}}\xspace}
\newcommand{\As} {\ensuremath{A_{\rm s}}\xspace}
%\newcommand{\nt} {\ensuremath{n_{\rm t}}\xspace}
%\newcommand{\At} {\ensuremath{A_{\rm t}}\xspace}
\newcommand{\thA}{\ensuremath{\theta_{\rm A}}\xspace}
%\newcommand{\run}{\ensuremath{{dn_s \over d\ln k}}\xspace}
\newcommand{\neff}{\ensuremath{N_\textrm{eff}}\xspace}
\newcommand{\mnu}{\ensuremath{\sum{m_{\nu}}}\xspace}
\newcommand{\yhe}{\ensuremath{Y_p}\xspace}
%\newcommand{\nrun}{\ensuremath{dn_s/d\ln k}\xspace}
\newcommand{\Map}[1]{\left<M^2_\textrm{ap}\right>( #1 )}
\newcommand{\map}{\ensuremath{\left<M^2_\textrm{ap}\right>}\xspace}
\newcommand{\chiH}{\ensuremath{\chi_\textrm{H}}\xspace}
\newcommand{\n}{\ensuremath{{\nu}\rm}\xspace}
\newcommand{\nue}{\ensuremath{{\nu}_{\rm e}}\xspace}
\newcommand{\num}{\ensuremath{{\nu}_{\rm \mu}}\xspace}
\newcommand{\nut}{\ensuremath{{\nu}_{\rm \tau}}\xspace}
\newcommand{\da}{\ensuremath{D_{\rm A}}\xspace}
\newcommand{\dl}{\ensuremath{D_{\rm L}}\xspace}
%\newcommand{\pripk}{\ensuremath{P_{\textrm{pri}}(k)}}
\newcommand{\taueq}{\ensuremath{\tau_{\rm eq}}\xspace}

%= = = = = = = = = = = = = = = = = = = = = = = = = = = = = = = = = = = = = = = =
% fancy header setup in P88
%\documentclass{book}
%\usepackage{fancyhdr}
%\pagestyle{fancy}
%% with this we ensure that the chapter and section
%% headings are in lowercase.
%\renewcommand{\chaptermark}[1]{%
%        \markboth{#1}{}}
%\renewcommand{\sectionmark}[1]{%
%        \markright{\thesection\ #1}}
%\fancyhf{}  % delete current header and footer
%\fancyhead[LE,RO]{\bfseries\thepage}
%\fancyhead[LO]{\bfseries\rightmark}
%\fancyhead[RE]{\bfseries\leftmark}
%\renewcommand{\headrulewidth}{0.5pt}
%\renewcommand{\footrulewidth}{0pt}
%\addtolength{\headheight}{0.5pt} % space for the rule
%\fancypagestyle{plain}{%
%   \fancyhead{} % get rid of headers on plain pages
%   \renewcommand{\headrulewidth}{0pt} % and the line
%}

