
\chapter{Improving the Source Plane Morphological Reconstruction of Multiply Imaged Galaxies}

\section{Gravitational lensing and source reconstruction techniques}\label{sec:model}

\subsection{Second order local corrections to the potential}

Given a cluster mass model with a gravitation potential $\psi^M$, the lens equation reads

\begin{equation}
\boldsymbol \beta^M=\boldsymbol \theta-\nabla \psi^M = \boldsymbol \theta - \boldsymbol \alpha^M,
\label{eq:lensequationM}
\end{equation}

where we use the superscript to indicate that quantities that depend on the model. The local distortion of an image is then given by the canonical relation

\begin{equation}
\frac{\partial \boldsymbol \beta_i^M}{\partial \boldsymbol \theta_j}=\delta_{ij}-\frac{\partial^2 \psi^M}{\partial \theta_i \partial \theta_j}=\left(\begin{array}{cc} 1-\kappa^M-\gamma_1^M & -\gamma_2^M \\
-\gamma_2^M & 1-\kappa^M+\gamma_1^M \end{array} \right) \equiv A_{ij}^M
\label{eq:magnification}
\end{equation}

However, the true potential will be in general different from the one estimated by the model:

\begin{equation}
\boldsymbol \beta=\boldsymbol \theta-\nabla \psi = \boldsymbol \theta - \boldsymbol \alpha,
\label{eq:lensequation}
\end{equation}

We now describe a procedure to obtain the first and derivatives of
$\delta \psi$. By subtracting Equation~\ref{eq:lensequationM}
from~\ref{eq:lensequation} we obtain

\begin{equation}
\boldsymbol \beta - \boldsymbol \beta^M=-\nabla \psi - \nabla \psi^M \equiv -\nabla \delta \psi = \boldsymbol \alpha - \boldsymbol \alpha^M.
\label{eq:lensequationc}
\end{equation}

Considering a reference image A, and a different image B, and assuming that $\psi^M(\boldsymbol \theta_A)$ is sufficiently accurate for our purposes (i.e $\delta \psi (\boldsymbol \theta_A)=0$ we can obtain the correction to the first derivative of the potential at position B, by computing equation \ref{eq:lensequationc} at A and B and equating $\beta$. Thus

\begin{equation}
\nabla \delta \psi(B) = \boldsymbol \beta_B^M - \beta_A^M.
\label{eq:deflectionc}
\end{equation}

The correction to the distortion matrix can be derived by considering equation~\ref{eq:lensequationc} in the form

\begin{equation}
\boldsymbol \beta = \boldsymbol \beta^M -\nabla \delta \psi.
\label{eq:lensequationc2}
\end{equation}

There are two alternatives. The first is to consider the transformation from
$\beta$ to $\theta$ directly, which gives rise to a symmetric matrix of easy interpretation:

%\begin{equation}
%\frac{\partial \boldsymbol \beta_i}{\partial \boldsymbol \beta_j^M}=\delta_{ij}-\frac{\partial^2 \delta \psi}{\partial \beta_i^M \partial \beta_j^M} \equiv \delta A_{ij}.
%\label{eq:magnificationc}
%\end{equation}

\begin{equation}
\frac{\partial \boldsymbol \beta_i}{\partial \boldsymbol \theta_j}=A^M_{ij}-\frac{\partial^2 \delta \psi}{\partial \theta_i \partial \theta_j} \equiv A^M_{ij}+\delta A_{ij} \equiv B_{ij}
\label{eq:magnificationc}
\end{equation}

The elements of $B_{ij}$ are obtained by optimizing the match between
image B and the reference image C (for example by matching the
position angle ellipticity and size or pixel by pixel). Note that this
matrix is symmetric and square by construction, and $\delta A_{ij}$
can be immediately interpreted as correction to the shear and
convergence.

Alternatively, one can write

\begin{equation}
\frac{\partial \boldsymbol \beta_i}{\partial \boldsymbol \theta_j}=
\sum_{k}\frac{\partial \boldsymbol \beta_i}{\partial \boldsymbol \beta^M_k}
\frac{\partial \boldsymbol \beta^M_k}{\partial \boldsymbol \theta_j}=
B_{ij}
\label{eq:magnificationm}
\end{equation}

switching to matrix notation and defining the additional matrix $\mathbf C$

\begin{equation}
\mathbf C  \mathbf A = \mathbf B,
\label{eq:matrix1}
\end{equation}

where

\begin{equation}
\mathbf C_{ij} = \frac{\partial \boldsymbol \beta_i}{\partial \boldsymbol \beta^M_j}
\label{eq:defC}
\end{equation}

And remembering that $\mathbf A$ is symmetric and invertible (non-zero determinant) we can write

\begin{equation}
\mathbf C = \mathbf B \mathbf A^{-1},
\label{eq:matrix1}
\end{equation}

I believe what you have derived in your algorithm is $\mathbb C$,
imposing that it is symmetric, which it needs not be [However:
physically, I am convinced it must be because of some symmetry in the
way you derive it, after all you can only align PA, ellipticity and
size, so three numbers, not 4!]. So you just need to multiply it by
$\mathbb A$ to obtain $\mathbb B$. Then

\begin{equation}
    \delta \mathbb A = \mathbb B - \mathbb A= \mathbb C \cdot \mathbb A - \mathbb A = \mathbb {(C-I)\cdot A} \neq 
    \mathbb {A\cdot (C-I)}
\label{eq:matrix1}
\end{equation}

which you can then interpret in terms of corrections to shear and
convergence. Also, det (CA)=det(C)det(A)=det(B), so the total
magnification is the product of the magnifications.

In practice, in order to interpret as shear and convergence:

\begin{equation}
\delta \kappa =
\frac{1}{2}(-\delta A_{11} - \delta A_{22}) =
\frac{1}{2} \mathbb A
\end{equation}

\begin{equation}
\delta \gamma_2 = -\delta A_{12}
\end{equation}

\begin{equation}
\delta \gamma_1 = \frac{1}{2} (-\delta A_{11} + \delta A_{22})
\end{equation}


