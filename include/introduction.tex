
\chapter{Introduction}

\section{What is the baryon cycle and why is it important?}

Baryon cycle is the cycling of the cosmic baryon budget through star formation and gas flows. Galaxies grow by
getting their fuel from the intergalactic medium (gas inflows), converting it into stars (star formation), and
returning energy/material to their surroundings (gas outflows). Figure~\ref{fig:M82} presents a vivid example of
this complex phenomenon in the local Universe.
%= = = = = = = = = = = = = = = = = = = = = = = = = = = = = = = = = = = = = = = =
%%% Figures: M82
\begin{figure}
    \centering
    \includegraphics[width=0.8\textwidth]{/Users/albert/Dropbox/latex_figures/schematics/M82_gasflows_texted.png}
    \caption[Onset of baryon cycling in the local Universe: M82.]{Onset of baryon cycling in the local Universe: the Messier 82
    galaxy. Overlaid are the conjectured movements of gas in and around this galaxy undergoing active star
    formation.
    \label{fig:M82}}
\end{figure}
%- - - - - - - - - - - - - - - - - - - - - - - - - - - - - - - - - - - - - - - -
While we have developed a concordance cosmological framework of the $\Lambda$ cold dark matter ($\Lambda$CDM)
hierarchical structure formation \citep[see \eg][and references therein]{PlanckCollaboration:2018va}, 
we still lack a quantitative and coherent understanding of these baryonic
processes, without which the formation and evolution of galaxies, e.g., our Milky Way (MW), cannot be
characterized in detail.

\section{How can we observe the baryon cycle at the peak of cosmic star formation?}

The integral field spectroscopy (IFS) has expanded our vision of galaxies from integrated spectroscopic
quantities extracted from small subregions through single slits/fibres to panoramic 2-dimensional (2D) views
across their full surfaces, allowing for spatial variations of physical properties.  This facilitates several
large ground-based surveys, \eg, MaGNA \citep{Bundy:2015ft,Law:2015hc}, SAMI
\citep{TheSAMIGalaxySurv:wpN2l47X,Medling:2018hx}, CALIFA \citep{Sanchez:2016cs}, to capture the dynamic
signatures of the baryon cycle in action.
%rich baryonic processes of galaxy assembly, feedback, and chemical enrichment in fine spatial richness.
Yet the requirement of sufficient sampling, \ie, on sub-kiloparsec (sub-kpc) scale, to accurately map interesting
kinematic features and spatial structures have limited the focus of such surveys to nearby galaxies
($z\lesssim0.03$).  However, the local Universe is only the downhill of cosmic star formation, and the true
climax of baryonic mass assembly and chemical enrichment happens at the epoch of $1\lesssim z\lesssim3$, as shown
in Figure~\ref{fig:cosmic_noon} \citep[see \eg, ][and references there in]{2014ARA&A..52..415M,Abramson:2016wf}.
%= = = = = = = = = = = = = = = = = = = = = = = = = = = = = = = = = = = = = = = =
%%% Figures
\begin{figure}
    \centering
    \includegraphics[width=.8\textwidth]{/Users/albert/Dropbox/latex_figures/othersPaper/abramson16_fig1_noon.png}
    \caption[The rise and fall of the cosmic average star-formation rate density.]{The rise and fall of the
    cosmic average star-formation rate density.
    The grey shaded region marks the peak epoch of cosmic star formation, usually coined the ``cosmic noon''.
    At the cosmic noon, our Universe assembled roughly one half of its present day stellar mass using less than one third of its age.
    Figure credit: \citet{Abramson:2016wf}.
    \label{fig:cosmic_noon}}
\end{figure}
%- - - - - - - - - - - - - - - - - - - - - - - - - - - - - - - - - - - - - - - -

Numerical hydrodynamic simulations are by design capable of reproducing realistic galaxies at this epoch, by
modeling various baryonic processes as sub-grid prescriptions
\citep{Dave:2017gd,FIRESimulations:2017wj,Pillepich:2019uz}. The vast majority of observational programs
targeting cosmic-noon galaxies only measure their integrated spectroscopic quantities through single slits.
Albeit offering great insight on the physics of galaxy formation and evolution at a population level, this kind
of integrated spectroscopy lacks the ability to dissect the baryon cycle in individual systems.
%However, there is ample evidence that the morphological transformation also occurs during the cosmic noon
%\citep{Mortlock:2013dg}, such that the predominant shape of galaxies change from irregular to ordered structure,
%\ie, disk and bulge, which have drastically different spectral properties. Neglecting this distinction might
%result in a biased picture of the baryon cycle.  In Fig.~\ref{fig:ers}, I demonstrate such bias associated with
%the derivation of key physical properties from spectroscopy with partial spatial coverage.
Ground-based IFS is no longer capable of probing the necessarily fine details at sub-kpc scale, ascribed to the
atmospheric beam smearing effect and the intrinsically small size of high-$z$ galaxies.  As displayed in
Figure~\ref{fig:spatial_resolution}, the half-light radius of a typical \Lstar galaxy at $z$$\sim$2 is
$\sim$2.5\kpc. In comparison, the optimal median seeing of $\sim$$0\farcs6$ is equivalent to a physical size of
$\sim$5\kpc, resulting in the vast majority of $z$$\sim$2 star-forming galaxies unresolved.  The assistance of
adaptive optics (AO) can mitigate this problem, yet is challenging due to several reasons: low Strehl ratio and
throughput, dearth of tip/tilt stars, narrow field of view with limited multiplexing capability, etc.  Therefore,
we need a more effective approach for sub-kpc scale spatially resolved spectroscopy of cosmic noon sources to
compare statistically with theoretical results of cosmological zoom-in simulations, to shed light upon the
cycling of baryons.

%= = = = = = = = = = = = = = = = = = = = = = = = = = = = = = = = = = = = = = = =
%%% Figures
\begin{figure}
    \centering
    \includegraphics[width=0.8\textwidth]{/Users/albert/Dropbox/latex_figures/schematics/spatial_resolution_difference_v0.png}
    \caption[The relationship among the effective radius, star-formation rate and stellar mass for galaxies at
    $z\sim2$.]{The relationship among the effective radius, star-formation rate and stellar mass for galaxies at $z\sim2$ covered by the CANDELS survey.
    The region where MW progenitors reside according to abundance matching is denoted by a red circle.
    The spatial resolution from natural seeing, native HST, and with gravitational lensing magnification combined
    are marked by the horizontal lines.
    We see that the only feasible approach to unambiguously resolve the MW progenitors is through the synergy of HST
    diffraction limit and lensing magnification.
    Figure credit: \citet{Wuyts:2012iw}.
    \label{fig:spatial_resolution}}
\end{figure}
%- - - - - - - - - - - - - - - - - - - - - - - - - - - - - - - - - - - - - - - -

%= = = = = = = = = = = = = = = = = = = = = = = = = = = = = = = = = = = = = = = = paragraph about GLASS and
%slitless grism spectroscopy from HST
In my dissertation, I took a different route.  The native resolution of the wide-field camera three (WFC3)
onboard the Hubble Space Telescope (\hst) is $\sim0\farcs13$, equivalent to a spatial scale of $\sim$1\kpc at
$z\sim2$ (cf.  Figure~\ref{fig:spatial_resolution}), already matching the spatial resolution achieved by the
aforementioned IFS surveys (\ie MaGNA, SAMI, CALIFA) in mapping the physical properties of nearby galaxies.  This
spatial sampling rate is unhampered by atmospheric turbulence because the telescope is floating on top.  The
near-infrared (NIR) spectroscopy on HST is enabled by the WFC3 grism elements.  During observation, light from the
science target is dispersed along a certain spatial axis pertinent to the telescope's roll angle \emph{without
any slits}.  As a consequence, source morphology along the light dispersion direction is convolved onto the
wavelength axis, causing the morphological broadening effect \citep{vanDokkum:2011cq}.  As shown in
Figure~\ref{fig:macs1149_grism}, using some state-of-the-art data reduction techniques, I have been able to
extract the appropriate 2D spectral information of my targets in spite of the heavy blending of neighboring
objects and morphological broadening effect, to facilitate subsequent spatially resolved analysis.

The gravitational lensing phenomenon further improves the spatial resolution by roughly $\sqrt{\mu}$,
where $\mu$ is the lensing magnification factor.
Thus by targeting cosmic-noon galaxies in the background and
gravitationally lensed by foreground clusters of galaxies with the \hst WFC3/NIR grism elements, I exploit the
lensing boost of \hst's resolving power to achieve sub-kpc scale spatial sampling of high-$z$ galaxies.
Therefore I compile a statistically significant sample of star-forming galaxies at cosmic noon,
whose spatial distribution of key physical properties are mapped out \emph{precisely}.
These properties include stellar mass (\Mstar), star-formation rate (SFR), gas-phase
metallicity\footnote{Throughout this dissertation I refer to the relative abundance of oxygen to hydrogen in the
interstellar medium as gas-phase metallicity, or just metallicity for simplicity.}, dust extinction, stellar
population age, total gas mass, etc.
This sample of galaxies has been shown to provide key insights into the effects of gas flows and feedback on the
cycling of baryons at the cosmic noon.

%= = = = = = = = = = = = = = = = = = = = = = = = = = = = = = = = = = = = = = = =
%%% Figures
\begin{figure}
    \centering
    \includegraphics[width=\textwidth]{/Users/albert/Dropbox/latex_figures/grismReduction/M1149_PA111_grismdemo.png}
    \caption[State-of-the-art data reduction techniques for \hst WFC3/NIR slitless spectroscopy.]{
    State-of-the-art data reduction techniques for \hst WFC3/NIR slitless spectroscopy are developed and adopted 
    throughout my dissertation work.
    \textit{Left}: the color-composite image of the galaxy cluster \clyi, where the first multiply imaged
    supernova (\ie SN Refsdal) is discovered, as marked by the white arrows \citep{2015Sci...347.1123K}.
    \textit{Middle}: the discovery of SN Refsdal triggers an HST follow-up grism program (Proposal ID 14041, P.I.
    Kelly) exposing the center of \clyi with 30 orbits of WFC3/G141 split equally into 2 position angles.
    The obtained grism exposure from one of them is shown here.
    The analysis of these data constitutes a significant part in my dissertation work (see Chapter 3).
    \textit{Right}: Due to the slitless nature of \hst grism spectroscopy, severe blending takes place for
    neighboring objects along the light dispersion direction (see the middle panel).
    The data reduction techniques used throughout my dissertation work (see Chapter 4 for more details) 
    properly account for this neighbour contamination and construct 2D spectral models for all sources 
    within the WFC3 field of view (FoV).
    These spectral models are refined iteratively until a convergence point where the residual in the grism
    exposures after subtracting the fitted continuum models becomes negligible, which is shown here.
    Figure credit: Gabe Brammer.
    \label{fig:macs1149_grism}}
\end{figure}
%- - - - - - - - - - - - - - - - - - - - - - - - - - - - - - - - - - - - - - - -


%= = = = = = = = = = = = = = = = = = = = = = = = = = = = = = = = = = = = = = = =
\section{Dissertation Overview}

The content of my dissertation is briefly described as follows.

Chapter 2 is adapted from my first-author paper published as Wang et al. (2015), ApJ, 811, 29, DOI:
10.1088/0004-637X/811/1/29, and is reproduced with permission from the AAS.
It presents a strong and weak lensing reconstruction of the massive cluster \cler, the first cluster for
which deep Hubble Frontier Fields (\hff) images and spectroscopy from the Grism Lens-Amplified Survey from Space
(\glass) are available.
By performing a targeted search for emission lines in multiply imaged sources using the
GLASS spectra, I obtain five high-confidence spectroscopic redshifts and two tentative ones. I confirm one
strongly lensed system by detecting the same emission lines in all three multiple images. I also search for
additional line emitters blindly and use the full GLASS spectroscopic catalog to test reliability of photometric
redshifts for faint line emitters. I see a reasonable agreement between our photometric and spectroscopic
redshift measurements, when including nebular emission in photometric redshift estimations. I introduce a
stringent procedure to identify only secure multiple image sets based on colors, morphology, and spectroscopy. By
combining 7 multiple image systems with secure spectroscopic redshifts (at 5 distinct redshift planes) with 18
multiple image systems with secure photometric redshifts, I reconstruct the gravitational potential of the
cluster pixellated on an adaptive grid, using a total of 72 images. The resulting mass map is compared with a
stellar mass map obtained from the deep \spitzer Frontier Fields data to study the relative distribution of stars
and dark matter in the cluster. I find that the stellar to total mass ratio varies substantially across the
cluster field, suggesting that stars do not trace exactly the total mass in this interacting system. 

Chapter 3 is adapted from my first-author paper published as Wang et al. (2017), ApJ, 837, 89, DOI:
10.3847/1538-4357/aa603c, and is reproduced with permission from the AAS.
I combine the deep \hst grism spectroscopy from both \glass and the SN Refsdal follow-up program (see
Figure~\ref{fig:macs1149_grism}) with a new Bayesian method to derive maps of gas-phase
metallicity for 10 star-forming galaxies at
high redshift ($1.2\lesssim z \lesssim2.3$). Exploiting lensing magnification by the foreground cluster \clyi, I
reach sub-kpc spatial resolution and push the limit of stellar mass associated with such high-z spatially
resolved measurements below $10^8$ \Msun for the first time. My maps exhibit diverse morphologies, indicative of
various effects such as efficient radial mixing from tidal torques, rapid accretion of low-metallicity gas, and
other physical processes that can affect the gas and metallicity distributions in individual galaxies. Based upon
an exhaustive sample of all existing sub-kpc resolution metallicity gradient measurements at high $z$, I find
that predictions given by analytical chemical evolution models assuming a relatively extended star-formation
profile in the early disk-formation phase can explain the majority of observed metallicity gradients, without
involving galactic feedback or radial outflows.  I observe a tentative correlation between stellar mass and
metallicity gradients, consistent with the ``downsizing'' galaxy formation picture that more massive galaxies are
more evolved into a later phase of disk growth, where they experience more coherent mass assembly at all radii
and thus show shallower metallicity gradients. In addition to the spatially resolved analysis, I compile a sample
of homogeneously cross-calibrated integrated metallicity measurements spanning three orders of magnitude in
stellar mass at $z\sim1.8$. We use this sample to study the mass-metallicity relation (MZR) and find that the
slope of the observed MZR can rule out the momentum-driven wind model at a 3-$\sigma$ confidence level.

Chapter 4 is adapted from my first-author paper in press, \ie, Wang et al. (2019) arXiv:1808.08800.
I report the first sub-kpc spatial resolution measurements of strongly inverted gas-phase 
metallicity gradients in two dwarf galaxies at $z$$\sim$2.
The galaxies have stellar masses $\sim$$10^9$\Msun, specific star-formation rate $\sim$20 \Gyr$^{-1}$, and 
global metallicity $\oh\sim8.1$ (1/4 solar).
Their metallicity radial gradients are measured to be highly inverted, \ie, 0.122$\pm$0.008 and 
0.111$\pm$0.017 dex/kpc, which is hitherto unseen at such small masses in similar redshift ranges.
From the HST observations of the source nebular emission and stellar continuum, I
present the 2D spatial maps of star-formation rate surface density, stellar population age, and 
gas fraction, which show that the two galaxies are currently undergoing rapid mass assembly via disk 
inside-out growth.  More importantly, using a simple chemical evolution model, I find that the gas 
fractions for different metallicity regions cannot be explained by pure gas accretion.  My spatially 
resolved analysis based on a more advanced gas regulator model results in a spatial map of net gaseous 
outflows, triggered by active central starbursts, that potentially play a significant role in shaping the 
spatial distribution of metallicity by effectively transporting stellar nucleosynthesis yields outwards.  
The relation between wind mass loading factors and stellar surface densities measured in different regions 
of our galaxies shows that a single type of wind mechanism, driven by either energy or momentum 
conservation, cannot explain the entire galaxy.  These sources present a unique constraint on the effects 
of gas flows on the early phase of disk growth from the perspective of spatially resolved chemical 
evolution within individual systems.

Chapter 5 is a summary of my previous work, concluding with a prospect of my ongoing analyses.


