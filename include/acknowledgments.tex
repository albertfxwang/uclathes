
First of all, I should thank my advisor Tommaso Treu, for his guidance, patience, and in particular his generous financial
support throughout the entire course of my PhD studies.
His ruthless pragmatism educated me of the urgency of publishing to avoid being replaced by collaborators.
I would also like to acknowledge my external advisor, Tucker Jones, who have helped me grow scientifically.
A significant portion of my PhD research is ``prototyped'' in his seminal work and I have been following his footsteps
ever since to have finally climbed over this monumental mountain called Doctor of Philosophy.

I am also greatly indebted to Louis Abramson. Tommaso and Tucker \emph{had} to deal with me, but for some reasons Louis \emph{chose} to.
His profound understanding of galaxy evolution combined with his eloquent arguments has always made our discussions highly
beneficial and enlightening to me.
An enormous thank you also goes to Takahiro Morishita.
He was my inspiration, as we face similar situations doing research abroad.
Being in a foreign country, where the speaking language does not resemble at all our mother tongues,
we can only earn respect through hard work, wholehearted dedication, and unique expertise.
In a way, Takahiro showed me a path and I am so proud that I have found the courage to follow it.

I would never take for granted the ample opportunities of close collaborations with experts in the field, who have
helped transform me into a confident and resourceful astronomer.
First and foremost, I would say millions of thank you to Gabe Brammer, for giving me the precious chance to come to STScI to
receive face-to-face hands-on training on grism data reduction from him, and answering me numerous questions on
astronomical data analysis in general.  ``Learning from Achilles himself; kings would kill for the honor.'' I am
sincerely grateful for having the great opportunity to collaborate with and learn from Emanuele Daddi.  He played
a crucial role in helping develop the inverted gradient discovery paper, and really expanded my horizon on the
physics of galaxy evolution. I cannot thank Keren Sharon enough for her help and advice, as well as giving me the
awesome chance to visit UMich.  She is the one who redeemed my faith in gravitational lensing
for galaxy clusters, by teaching me Lenstool step-by-step, which now has become an indispensible part in my skill
set.  Matt Malkan is greatly appreciated for his pivotal help in securing a postdoc position for me at
Caltech/IPAC starting this summer, and for allowing me to join the WISPs collaboration.  I acknowledge Kasper
Schmidt and Kuang-Han Huang, who initiated me on Python programming.

I would also like to thank Omer Blaes, Matt Malkan, Alice Shapley and Tommaso Treu for serving on my dissertation
committee.

My deep gratitude also goes to all my fellow graduate students at both UCSB and UCLA, in particular, Suoqing Ji,
Xinnan Du, and Ryan Sanders.
Suoqing started graduate school at UCSB in the same year as I did, and has given me some great tips in
transitioning to US life.
Becoming close friends with Suoqing is one of the few things that made my days at UCSB not a total failure.
Xinnan helped me a great deal with my transfer from UCSB to UCLA.
Without her invaluable help, I could not have won the prestigious Chinese Government Award for Outstanding
Graduate Student Abroad.
Ryan served as my graduate student mentor for over a year, and set up an excellent role model for me.
I sincerely thank him for proof-reading my proposals/papers numerous times.
Also acknowledged are the Treu group members whom I have been sharing pizzas with:
Alessandro Sonnenfeld, Simon Birrer, Peter Williams, Anowar Shajib, Daniel Gilman, Xuheng Ding, Lilan Yang.

I would also like to appreciate the kindness and assistance from local people when I visited their institutes.
First of all, Prof. Shude Mao is wholeheartedly acknowledged for giving me multiple opportunities of lengthy visits to THCA.
Bravo on the recently founded Department of Astronomy.
I am also truly grateful to the help and financial support from Prof. Le Zhang for my visit to SJTU, 
Prof. Xu Kong for my visit to USTC, Prof. Yu Gao for my visit to PMO, Prof. Gongbo Zhao for my visit to NAOC.
Furthermore, thank you also goes to Jessie Hirtenstein at UC Davis,
Raymond Simons, Alaina Henry and Ivelina Momcheva at STScI,
Yuan-Sen Ting at IAS,
Jacqueline van Gorkom at Columbia,
Xiangcheng Ma, Phil Hopkins, Nicha Leethochawalit and Evan Kirby at Caltech,
Xiangcheng Ma (again), Yuan Li, and Dan Weisz at UC Berkeley,
Zheng Cai, Joseph Burchett and Song Huang at UCSC.
Since my college day one, I have been living in the giant shadow shaded by Song; 
his fanatic passion for astronomy is extremely contagious and has always spurred me onto greater effort in
pursuit of astronomical achievements.
My special gratitude to Hui Li, whos is also a Nanjing University alumnus.
He provided tremendous help to me when I visited Ann Arbor as well as Boston.
Congratulations to his recent achievement on the Hubble fellowship.
Last but not the least, the professors at Nanjing University, i.e., Profs. Zhiyuan Li, Yong Shi, Qiu-Sheng Gu,
Yong Feng Huang, Jilin Zhou, etc., are cordially acknowledged for my several trips back to my Alma Mater.
It is always a mind-blowing pleasure to see that Astronomy at NJU has expanded on such a large scale in recent
years and I look forward to its everlasting growth.

Last but not the least, without my parents, I could not have been here in the first place. Their care and
support, both financially and emotionally, are the cornerstones of every piece of my achievements, ever since I
was a little kid.
My last and best spot is always reserved for the love of my life, Xiaolei!
But nothing written down could come even close to do justice to your love and what it means to me.
Earning a PhD is a tumultuous and hectic journey.
Without you, I would have long been lost.

