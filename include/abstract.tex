To explore the chemo-structural properties of galaxies and understand quantitatively the cycling of baryons at
the peak epoch of cosmic star formation, I developed a highly effective method for sub-kiloparsec scale spatially
resolved spectroscopy of strongly lensed galaxies using space-based wide-field slitless grism data.  Applying
this method to the deep Hubble Space Telescope (HST) near-infrared grism observations, I obtained precise
gas-phase metallicity maps for a large sample of star-forming galaxies in the redshift range of $1.2\lesssim
z\lesssim2.3$. Over
half of the galaxies in my sample reside in the dwarf mass regime ($\Mstar\lesssim10^9 \Msun$), making my sample the first
statistically representative sample of high-redshift dwarf galaxies with their metallicity spatial distribution
measured with sufficient resolution.  The metallicity maps obtained in my work reveal a variety of baryonic
physics, such as efficient radial mixing from tidal torques, rapid accretion of low-metallicity gas, and various
feedback processes which can significantly influence the chemo-structural properties of star-forming galaxies.
For the first time, I discovered two dwarf galaxies at $z\sim2$ displaying strongly inverted metallicity radial
gradients, suggesting that powerful galactic winds triggered by central starbursts carry the bulk of stellar
nucleosynthesis yields to the outskirts.  I also observed an intriguing correlation between stellar mass and
metallicity gradient, consistent with the ``downsizing'' galaxy formation picture that more massive galaxies are
more evolved into later phases of disk growth, where they experience more coherent mass assembly at all radii
and thus show shallower metallicity gradients.  Furthermore, 10\% of the metallicity gradients measured in my
sample are inverted, which are hard to explain by currently existing hydrodynamical simulations and analytical
chemical evolution models.  My method can be readily applied to data from future space missions employing grism
instruments, e.g., JWST, Euclid, WFIRST.  Combined with the continuous input of HST resources, these data will
revolutionize our understanding of the chemo-structural evolution of galaxies throughout vast cosmic time.
