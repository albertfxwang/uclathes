
To explore the chemo-structural properties of galaxies at cosmic noon (i.e. z~2), I developed a highly effective
method for sub-kiloparsec scale spatially resolved spectroscopy of strongly lensed galaxies using space-based
wide-field slitless grism data. Applying this method to the deep Hubble Space Telescope near-infrared grism data,
I obtained precise gas-phase metallicity maps of 81 star-forming galaxies at z~1.2-2.3, over half of which reside
in the dwarf mass regime. My work presents the first statistically representative sample of high-z dwarf galaxies
with their metallicity spatial distribution measured with sufficient resolution. These metallicity maps reveal a
variety of baryonic physics, such as efficient radial mixing from tidal torques, rapid accretion of
low-metallicity gas, and various feedback processes which can significantly influence the chemo-structural
properties of dwarf galaxies. In particular, we find two galaxies at z~2 displaying strongly inverted metallicity
radial gradients, suggesting that powerful galactic winds triggered by central starbursts carry the bulk of
stellar nucleosynthesis yields to the outskirts. Furthermore, 10\% of the metallicity gradients measured in our
sample are inverted, which are hard to explain by currently existing hydrodynamical simulations and analytical
chemical evolution models. My method can also be readily applied to data from future space missions employing
grism instruments, e.g., JWST, WFIRST.

