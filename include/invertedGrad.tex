
\chapter{Discovery of Strongly Inverted Metallicity Gradients in Dwarf Galaxies at $z$$\sim$2}

\section{Introduction}\label{sect:intro}

Galaxy formation models require inflows and outflows of gas to regulate star formation 
\citep{2008MNRAS.385.2181F,Recchi:2008gw,Bouche:2010kh,2012MNRAS.421...98D,Dayal:2013im,Dekel:2013id,Lilly:2013ko,Dekel:2014jm,Peng:2014hn,Pipino:2014it}, 
yet this ``baryon cycle'' is not quantitatively understood. The interstellar medium (ISM) oxygen abundance 
(\ie metallicity\footnote{Throughout the paper, we refer to \gpm as metallicity for simplicity.}) and its 
spatial distribution
is fortunately a key observational probe of this process
\citep{Tremonti:2004ed,Erb:2006kn,2008A&A...488..463M,Bresolin:2009hh,2010MNRAS.408.2115M,Mannucci:2011be,Zahid:2011bb,Yates:2012kx,Zahid:2012cd,
Henry:2013cn,2013ApJ...765...48J,2014A&A...563A..49S,TheUniversalRelati:2014kx,Bresolin:2015fk,Ho:2015gq,Sanders:2015gk,Strom:2016vn}.
``Inside-out'' galaxy growth implies that initially steep radial gradients of metallicity flatten at later 
times (higher masses)
as disks grow larger, yet other scenarios suggest metallicities are initially well mixed by strong galactic feedback, and then
locked into negative gradients as winds lose the power to disrupt massive gas disks
\citep{Prantzos:2000gb,Hou:2000tq,Molla:2005eq,Kobayashi:2011cr,Few:2012jl,Pilkington:2012ib,Gibson:2013jw,2017MNRAS.466.4780M}.
What in common between these scenarios is that none of them predict the existence of a steep positive (\ie 
inverted) radial gradient such that metallicity increases with galacto-centric radius.

However, there is growing evidence of such phenomenon in both the local and distant Universe
\citep{Cresci:2010hr,Queyrel:2012hw,2014MNRAS.443.2695S,Metallicityevolutio:2014kg,2014A&A...563A..49S,PerezMontero:2016hs,2016ApJ...827...74W,Belfiore:2017bv,Carton:2018kv}.
The key reason for local galaxies possessing inverted gradients is gas re-distribution by tidal force in strongly interacting
systems \citep{Kewley:2006gb,Kewley:2010eg,Rupke:2010cg,AnIntegralFieldSt:2012hn,Torrey:2012kf}.
At high redshifts, inverted gradients are often attributed to the inflows of metal-poor gas from the filaments of cosmic web,
infalling directly onto galaxy centers, diluting central metallicities and hence creating positive gradients
\citep{Cresci:2010hr,Mott:2013bt}.
Given most of the high-$z$ observations are conducted from the ground with natural seeing, the targets are usually
super-\Lstar galaxies with stellar mass (\Mstar) $\gtrsim$$10^{10}$\Msun \citep[see \eg,][]{Metallicityevolutio:2014kg}.

These high-$z$ inverted gradients are in concert with the ``cold-mode'' gas accretion which has long been recognized to play a
crucial role in galaxies getting their baryonic mass supply
\citep{Birnboim:2003fo,Keres:2005gb,Dekel:2006cn,Dekel:2009fz,2009MNRAS.395..160K}.
Instead of being shock-heated to dark matter (DM) halo virial temperature ($\sim$10$^6$K for a $M_{
h}$$\sim$10$^{12}$\Msun halo) and then radiate away the thermal energy to condense and form stars (\vsv 
``hot-mode'' accretion),
gas streams can remain relatively cold (<10$^5$K) while being steadily accreted onto galaxy 
disks\footnote{Note however that
cold-mode accretion does not necessarily enforce that gas has to reach galaxy center first
given the large dynamic range of the scales of galaxy disks ($\sim$kpc) and cosmic web ($\sim$Mpc).}.
This cold accretion dominates the growth of galaxies forming in low-mass halos irrespective of redshifts since a hot permeating
halo of virialized gas can only manifest in halos above 2-3$\times10^{11}$\Msun, at $z\lesssim2$
\citep{Birnboim:2003fo,Keres:2005gb}.

\subsection{Conclusion}

A question thus arises: if cold-mode gas accretion dominates in low-mass systems (with \Mstar less than a few
$10^{10}$\Msun) and is thought to lead to inverted gradients under the condition that the incoming gas streams 
are centrally directed, can we observe this phenomenon in dwarf galaxies (with $\Mstar\lesssim10^9$) at high 
redshifts?
The answer is not straightforward since the effect of ejective feedback (\eg galactic winds driven by supernovae) is more
pronounced in lower mass galaxies, given their shallower gravitational potential wells and higher specific
star-formation rate (sSFR) \citep[see \eg][]{GalaxiesonFIREFe:2014dn,2014Natur.509..177V}.
On one hand, galactic winds can bring about kinematic turbulence that prevents a smooth accretion of 
filamentary gas streams directly onto galaxy center, resulting in rapid formation of in-situ clumps 
\citep{Dekel:2009bn}.
On the other hand, metal-enriched outflows triggered by these powerful winds can help remove stellar nucleosynthesis yields from
galaxy center \citep{Tremonti:2004ed,Erb:2006kn}.
Therefore the existence of strongly inverted gradients in dwarf galaxies at high redshifts, if any, presents a sensitive test of 
the relative strength of feedback-induced radial gas flows, in the early phase of the disk mass assembly process.
There have not been any attempts to investigate such existence, primarily due to the small sizes of these 
dwarf galaxies and sub-kiloparsec (sub-kpc) spatial resolution required to yield accurate gradient 
measurements \citep{2013ApJ...767..106Y}.
In this work, we present the first effort to secure two robustly measured inverted metallicity gradients in $z\sim2$ star-forming
dwarf galaxies from the Hubble Space Telescope (\hst) near-infrared (NIR) grism slitless spectroscopy, aided 
with galaxy cluster lensing magnification.
The details of data and sample galaxies are presented in Section~\ref{sect:data}. We describe our analysis methods alongside main 
results in Section~\ref{sect:rslt}, and conclude in Section~\ref{sect:conclu}.
Throughout this paper, a flat $\Lambda$CDM cosmology is assumed.



