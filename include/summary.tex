
\chapter{Summary and Ongoing Work}

In this dissertation, I have presented several published work of mine focused on the following topics.

\begin{enumerate}
    \item I modeled the total mass distribution of galaxy clusters (\cler in particular) to calibrate the
    properties of these cosmic telescopes which magnify background sources to facilitate enhanced spatial
    sampling at sub-kpc resolution further improved over \hst's native diffraction limit.
    \item I devised a new Bayesian inference method to estimate gas-phase metallicity directly from strong
    nebular emission line fluxes rather than flux ratios, and brought forward a sample of ten star-forming
    galaxies at $z\sim2$ whose radial gradients of metallicity are measured at sub-kpc resolution, from the
    ultra deep \hst WFC3/NIR grismd observations.
    \item I reported the first ever measurements of strongly inverted metallicity gradients in dwarf galaxies
    ($\Mstar\approx10^9\Msun$) at $z\sim2$, and mapped out their net gaseous flows which indicate that
    feedback-triggered metal-enriched outflows transport stellar nucleosynthesis yields outwards thus
    inverting metallicity gradient.
\end{enumerate}

These work demonstrate the unique capability of the synergy of \hst WFC3/NIR slitless spectroscopy and
gravitational lensing in pinpointing the \emph{precise} spatial distribution of galactic chemical properties at
cosmic noon.  Combining these accurately mapped metallicity spatial distribution with the empirical modeling
framework of galaxy evolution and star formation (\eg chemical evolution models, gas regulator models, the KS
law, \etc), we are able to dissect the complex phenomena of galactic feedback and gas flows at the peak epoch of
cosmic baryonic mass assembly.

With that as the very goal, I am currently analyzing the entire \glass dataset, to put forward an unprecedentedly
large sample of galaxies in the redshift range of $1.2\lesssim z\lesssim2.3$, with precisely measured metallicity
gradients. The ten galaxy clusters targeted by \glass are shown in Table~\ref{tab:full_clusters}, and the
color-composite images of nine of them, synthesized from \hst broad-band imaging, are displayed in
Figure~\ref{fig:full_clusters}. Note that the RGB image of the remaining cluster (\ie \clyi) is already given in
Figure~\ref{fig:RGBfullFoV}.

\begin{landscape}
%= = = = = = = = = = = = = = = = = = = = = = = = = = = = = = = = = = = = = = = =
% = = = = = = = = = = = = = = = = = = = = = = = = = = = = = = = = = = = = = = = = = =
% Include this table with \input{filename.tex}
% To rotate in emulateapj do: \begin{turnpage}\input{filename.tex}\end{turnpage}
% To display it on multiple pages do: \LongTables\input{filename.tex}
% - - - - - - - - - - - - - - - - - - - - - - - - - - - - - - - - - - - - - - - - - -
\begin{deluxetable}{llcccclcccccc}
    \tablecolumns{13}
    \tablewidth{0pt}
    \tablecaption{Properties of the grism spectroscopic data used in this work}
% - - - - - - - - - - - - - - - - - - - - - - - - - - - - - - - - - - - - - - - - - -
\tablehead{
    \colhead{Cluster} &
    \colhead{Alias} &
    \colhead{Redshift} &
    \colhead{RA}  &
    \colhead{DEC} &
    \colhead{Grism P.A.s\tablenotemark{a}} & 
    \colhead{\hst imaging} \\
    & & &
    \colhead{(J2000)} &
    \colhead{(J2000)} &
    & 
}
%---------------------------------------------------------------
\startdata
    MACS0717.5+3745 & MACS0717  &   0.548   &   07:17:34.0  &  +37:44:49.0  & 020, 280 & \clash/\hff  \\
    MACS1423.8+2404 & MACS1423  &   0.545   &   14:23:48.3  &  +24:04:47.0  & 008, 088 & \clash   \\
    MACS1149.6+2223\tablenotemark{b} & MACS1149  &   0.544   &   11:49:36.3  &  +22:23:58.1  & 032, 111, 119, 125 & \clash/\hff &  \\
    RXJ1347.5-1145  & RXJ1347   &   0.451   &   13:47:30.6  &  -11:45:10.0  & 203, 283 & \clash   \\
    RXJ2248.7-4431  & RXJ2248   &   0.348   &   22:48:44.4  &  -44:31:48.5  & 053, 133 & \clash/\hff  \\
    MACS2129.4-0741 & MACS2129  &   0.570   &   21:29:26.0  &  -07:41:28.0  & 050, 328 & \clash   \\
    Abell 2744      & A2744     &   0.308   &   00:14:21.2  &  -30:23:50.1  & 135, 233 & \hff   \\
    MACS0744.9+3927 & MACS0744  &   0.686   &   07:44:52.8  &  +39:27:24.0  & 019, 104 & \clash   \\
    Abell 370       & A370      &   0.375   &   02:39:52.9  &  -01:34:36.5  & 155, 253 & \hff   \\
    MACS0416.1-2403 & MACS0416  &   0.420   &   04:16:08.9  &  -24:04:28.7  & 164, 247 & \clash/\hff 
\enddata
% - - - - - - - - - - - - - - - - - - - - - - - - - - - - - - - - - - - - - - - - - -
    \tablenotetext{a}{The position angles (P.A.s) shown here correspond to the ``PA\_V3'' value reported in the WFC3/IR
    image headers. The position angle of the dispersion axis of the grism spectra is given by $\mathrm{PA_{disp}}
    \approx \mathrm{PA\_V3} - 45$.}
    \tablenotetext{b}{The data analysis for this field has been presented in \citet{Wang:2016um} (see Chapter~3).}
\label{tab:full_clusters}
\end{deluxetable}

%= = = = = = = = = = = = = = = = = = = = = = = = = = = = = = = = = = = = = = = =
\end{landscape}

Following similar data reduction and analysis procedures described in previous chapters, I compiled a sample of 81
galaxies, whose spatial distributions of \gpm are mapped out at sub-kpc resolution. These sources are marked by
magenta circles in Figure~\ref{fig:full_clusters}.
Collecting all currently existing sub-kpc resolution metallicity gradient measurements in high-$z$ star-forming
galaxies, I investigate the redshift evolution and mass dependence of metallicity gradients, as shown in
Figure~\ref{fig:full_metalgrad}.
Undoubtedly, the vast majority of these measurements come from my space-based slitless spectroscopy analysis.
Furthermore, this exhaustive sample of \emph{precise} metallicity gradients also includes measurements from ground-based AO-supported IFS observations \citep{2012MNRAS.426..935S,2013ApJ...765...48J,2015arXiv150901279L}. 
In comparison, ground-based seeing-limited measurements lack sufficient resolution due to beam smearing, 
and thus tend to show flat radial gradients (see the spread of \kd results, \citet{2016ApJ...827...74W}).

\begin{figure}
    \centering
    \includegraphics[width=\textwidth]{/Users/albert/data/abell2744_HFF/rgb.pdf}\\
    \caption[\glass grism exposures and source selection in this work.]
    {The color-composite images of the galaxy cluster centers exposed by \hst WFC3/NIR grism elements (10 orbits
    of G102 and 4 orbits of G141) from the \glass program. The entire amount of exposure time is equally
    deposited into 2 separate pointings represented by the red and green squares with light dispersion directions
    denoted by the arrows in the upper right corner.
    The sample of star-forming galaxies in which I secure \emph{unbiased} measurements of metallicity radial gradients are marked by magenta circles.
    The cyan contours are the critical curves at sample median redshift.
    }
    \label{fig:full_clusters}
\end{figure}

\begin{figure}
    \centering
    \includegraphics[width=.8\textwidth]{/Users/albert/data/macs0416_HFF/rgb.pdf}\\
    \includegraphics[width=.8\textwidth]{/Users/albert/data/macs0717_HFF/rgb.pdf}
    \contcaption{(cont.)}
\end{figure}

\begin{figure}
    \centering
    \includegraphics[width=.8\textwidth]{/Users/albert/data/abell370_HFF/rgb.pdf}\\
    \includegraphics[width=.8\textwidth]{/Users/albert/data/rxj2248_HFF/rgb.pdf}
    \contcaption{(cont.)}
\end{figure}

\begin{figure}
    \centering
    \includegraphics[width=.8\textwidth]{/Users/albert/data/rxj1347_CLASH/rgb.pdf}\\
    \includegraphics[width=.8\textwidth]{/Users/albert/data/macs0744_CLASH/rgb.pdf}
    \contcaption{(cont.)}
\end{figure}

\begin{figure}
    \centering
    \includegraphics[width=.8\textwidth]{/Users/albert/data/macs2129_CLASH/rgb.pdf}\\
    \includegraphics[width=.8\textwidth]{/Users/albert/data/macs1423_CLASH/rgb.pdf}
    \contcaption{(cont.)}
\end{figure}

With the sample statistics improved by one order of magnitude primarily thanks to my ongoing efforts, we start to
develop some crucial insights from the redshift and mass dependences of metallicity gradients.
The radial gradient of metallicity has been used as a great proxy of the strength and effectiveness of
galactic feedback. The orange tracks in the upper panel of Figure~\ref{fig:full_metalgrad} correspond to two
simulation realizations of MW at $z=0$, with different feedback sub-grid prescriptions but otherwise identical
numerical setup \citep[][G13]{Gibson:2013jw}.
This demonstrates that enhanced feedback can be really efficient in erasing metal inhomogeneity, and resolved
chemical properties measured in the early phase of disk growth can really shed light on the strength of galactic
feedback.

As a whole, the current measurements of \emph{unbiased} metallicity gradients (points in Figure~\ref{fig:full_metalgrad}) lead us to the following conclusions.
\begin{enumerate}
    \item The metallicity inhomogeneity throughout the disk mass assembly process is frequently seen, which is
    in acute disapproval of the constant influence of strong galactic feedback and/or galaxy coalescence
    (see the 2-$\sigma$ shaded region of Illustris simulations).
    \item There exist sizable scatters in the observational measurements of metal gradients, not yet reproduced
    in any single suites of cosmological hydro-simulations (see the 2-$\sigma$ shaded regions of IllustrisTNG and
    FIRE). But the FIRE simulation predicts the mass dependence of metallicity gradients in better agreement with
    observations.
    \item There is a non-negligible fraction (10/81) of galaxies showing inverted metallicity gradients. This
    phenomenon is quite unconventional from the classic theory of galaxy evolution and thus poses great
    challenge to the predicted evolutionary trends given by analytical chemical evolution models (see \eg, the
    shaded region of \citet{Molla:2005eq}).
\end{enumerate}

% a way forward and future prospect
The mass dependence of metallicity gradient encodes key information about the temporal evolution of galactic
chemo-structural properties.
If we could isolate cohorts of galaxies at different cosmic epochs (\ie with different masses) that 
hypothetically follow the same evolutionary trajectory to become MWs at $z=0$,
then the cross-sectional snapshots of these metallicity spatial distributions at various mass assembly stages
reflect the longitudinal behaviors of MW progenitors at the corresponding ages.
This approach can be extremely powerful in unraveling the chemical evolution of galaxies, and therefore cast 
strong constraints on the effect of star formation and galactic feedback in shaping the galactic chemo-structures
at various stages of disk growth.
Apparently it will require a much larger sample of galaxies targeted by diffraction-limited spatially resolved
spectroscopy, only achievable through space-based slitless spectroscopic campaigns.
With the continued input of \hst resources, as well as the great prospect promised by future space missions with
grism capabilities, \eg, the James Webb Space Telescope (JWST), the Euclid telescope, the Wide Field Infrared
Survey Telescope (WFIRST), I am confident that this much larger
sample will be obtained and will completely revolutionize our understanding of the chemo-structural evolution of 
galaxies (including our MW) throughout vast cosmic time.

\begin{figure}
    \centering
    \includegraphics[width=\textwidth]{/Users/albert/Dropbox/Python/metalP3/oh12gradVSz_plaskett.png}\\
    \includegraphics[width=\textwidth]{/Users/albert/Dropbox/Python/metalP3/oh12gradVSmstar_plaskett.png}
    \caption[The redshift evolution and mass dependence of all \emph{unbiasedly} measured metallicity gradients
    in high-$z$ star-forming galaxies.]
    {The redshift evolution and mass dependence of all metallicity gradients in high-$z$ star-forming galaxies 
    measured at sub-kpc resolution, unbiased from the beam smearing effect. It is obvious that my analysis results based on the
    combination of \hst WFC3/NIR grism slitless spectroscopy and lensing magnification constitute a major portion
    of all these measurements.}
    \label{fig:full_metalgrad}
\end{figure}

